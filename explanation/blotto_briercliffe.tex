%%%%%%%%%%%%%%%%%%%%%%%%%%%%%%%%%%%%%%%%%
% Structured General Purpose Assignment
% LaTeX Template
%
% This template has been downloaded from:
% http://www.latextemplates.com
%
% Original author:
% Ted Pavlic (http://www.tedpavlic.com)
%
% Note:
% The \lipsum[#] commands throughout this template generate dummy text
% to fill the template out. These commands should all be removed when 
% writing assignment content.
%
%%%%%%%%%%%%%%%%%%%%%%%%%%%%%%%%%%%%%%%%%

%----------------------------------------------------------------------------------------
%	PACKAGES AND OTHER DOCUMENT CONFIGURATIONS
%----------------------------------------------------------------------------------------

\documentclass{article}

\usepackage{fancyhdr} % Required for custom headers
\usepackage{lastpage} % Required to determine the last page for the footer
\usepackage{extramarks} % Required for headers and footers
\usepackage{graphicx} % Required to insert images
\usepackage{amsmath}
\usepackage{graphicx}
\usepackage{amssymb}
\usepackage{amsfonts}
\usepackage{hyperref}
\usepackage{url}
\usepackage{xcolor,colortbl}


% Margins
\topmargin=-0.45in
\evensidemargin=0in
\oddsidemargin=0in
\textwidth=6.5in
\textheight=9.0in
\headsep=0.25in 

\linespread{1.1} % Line spacing

% Set up the header and footer
\pagestyle{fancy}
\lhead{\hmwkAuthorName} % Top left header
\chead{\hmwkClass } % Top center header
\rhead{\quad \hmwkTitle} % Top right header
\lfoot{\lastxmark} % Bottom left footer
\cfoot{} % Bottom center footer
\rfoot{Page\ \thepage\ of\ \pageref{LastPage}} % Bottom right footer
\renewcommand\headrulewidth{0.4pt} % Size of the header rule
\renewcommand\footrulewidth{0.4pt} % Size of the footer rule

\setlength\parindent{0pt} % Removes all indentation from paragraphs

%----------------------------------------------------------------------------------------
%	DOCUMENT STRUCTURE COMMANDS
%	Skip this unless you know what you're doing
%----------------------------------------------------------------------------------------

\newcommand{\mc}[2]{\multicolumn{#1}{c}{#2}}
\definecolor{Gray}{gray}{0.85}
\definecolor{LightCyan}{rgb}{0.88,1,1}

\newcolumntype{a}{>{\columncolor{white}}c}
\newcolumntype{b}{>{\columncolor{white}}c}

% Header and footer for when a page split occurs within a problem environment
\newcommand{\enterProblemHeader}[1]{
\nobreak\extramarks{#1}{#1 continued on next page\ldots}\nobreak
\nobreak\extramarks{#1 (continued)}{#1 continued on next page\ldots}\nobreak
}

% Header and footer for when a page split occurs between problem environments
\newcommand{\exitProblemHeader}[1]{
\nobreak\extramarks{#1 (continued)}{#1 continued on next page\ldots}\nobreak
\nobreak\extramarks{#1}{}\nobreak
}

\setcounter{secnumdepth}{0} % Removes default section numbers
\newcounter{homeworkProblemCounter} % Creates a counter to keep track of the number of problems

\newcommand{\homeworkProblemName}{}
\newenvironment{homeworkProblem}[1][Problem \arabic{homeworkProblemCounter}]{ % Makes a new environment called homeworkProblem which takes 1 argument (custom name) but the default is "Problem #"
\stepcounter{homeworkProblemCounter} % Increase counter for number of problems
\renewcommand{\homeworkProblemName}{#1} % Assign \homeworkProblemName the name of the problem
\section{\homeworkProblemName} % Make a section in the document with the custom problem count
\enterProblemHeader{\homeworkProblemName} % Header and footer within the environment
}{
\exitProblemHeader{\homeworkProblemName} % Header and footer after the environment
}

\newcommand{\problemAnswer}[1]{ % Defines the problem answer command with the content as the only argument
\noindent\framebox[\columnwidth][c]{\begin{minipage}{0.98\columnwidth}#1\end{minipage}} % Makes the box around the problem answer and puts the content inside
}

\newcommand{\homeworkSectionName}{}
\newenvironment{homeworkSection}[1]{ % New environment for sections within homework problems, takes 1 argument - the name of the section
\renewcommand{\homeworkSectionName}{#1} % Assign \homeworkSectionName to the name of the section from the environment argument
\subsection{\homeworkSectionName} % Make a subsection with the custom name of the subsection
\enterProblemHeader{\homeworkProblemName\ [\homeworkSectionName]} % Header and footer within the environment
}{
\enterProblemHeader{\homeworkProblemName} % Header and footer after the environment
}
   
%----------------------------------------------------------------------------------------
%	NAME AND CLASS SECTION
%----------------------------------------------------------------------------------------

\newcommand{\hmwkTitle}{ \url{http://stat.ubc.ca/\textasciitilde creagh/}} % Assignment title
\newcommand{\hmwkDueDate}{Friday,\ September\ 29,\ 2017 } % Due date
\newcommand{\hmwkClass}{\texttt{creagh@stat.ubc.ca}} % Course/class
\newcommand{\hmwkClassName}{Social \& Information Networks} %Course name
\newcommand{\hmwkClassTime}{} % Class/lecture time
\newcommand{\hmwkClassInstructor}{} % Teacher/lecturer
\newcommand{\hmwkAuthorName}{Creagh Briercliffe} % Your name
\newcommand{\hmwkStudentNum}{78402147}% My student number
\newcommand{\hmwkEmail}{creagh@stat.ubc.ca} %My email

%----------------------------------------------------------------------------------------
%	TITLE PAGE
%----------------------------------------------------------------------------------------

\title{
\vspace{2in}
\textmd{\textbf{\hmwkClass:\ \hmwkClassName}}\\
\textmd{\textbf{\hmwkTitle}}\\
\normalsize\vspace{0.1in}\small{Due\ on\ \hmwkDueDate}\\
%\vspace{0.1in}\large{\textit{\hmwkClassInstructor}}
\vspace{3in}
}

\author{
\textbf{\hmwkAuthorName}\\
\textbf{\hmwkStudentNum}\\
\textbf{\hmwkEmail} 
}

\date{} % Insert date here if you want it to appear below your name

%----------------------------------------------------------------------------------------

\begin{document}

%\maketitle

%----------------------------------------------------------------------------------------
%	TABLE OF CONTENTS
%----------------------------------------------------------------------------------------

%\setcounter{tocdepth}{1} % Uncomment this line if you don't want subsections listed in the ToC

%\newpage
%\tableofcontents
%\newpage


\centerline{\LARGE\textbf{Jane Street Symposium - Blotto}}

%----------------------------------------------------------------------------------------
%	PROBLEM 1
%----------------------------------------------------------------------------------------

% To have just one problem per page, simply put a \clearpage after each problem

\begin{homeworkProblem}[\arabic{homeworkProblemCounter}. What's your entry?]
%\includegraphics[width=0.75\columnwidth]{example_figure} % Example image

\begin{table}[h]
\center
\begin{tabular}{| l | a | b | a | b | a | b | a | b | a | b |}
\hline
\rowcolor{LightCyan}
District  & D$_1$ & D$_2$ & D$_3$ & D$_4$ & D$_5$ & D$_6$ & D$_7$ & D$_8$ & D$_9$ & D$_{10}$\\
\hline
My Entry & 37 & 3 & 4 & 5 & 6 & 7 & 8 & 9 & 10 & 11 \\ \hline
\end{tabular}
\end{table}

\end{homeworkProblem}

%----------------------------------------------------------------------------------------
%	PROBLEM 2
%----------------------------------------------------------------------------------------

\begin{homeworkProblem}[\arabic{homeworkProblemCounter}. How did you go about coming up with it?]

\textbf{High-level idea:}
\\

I simulated several of my own tournaments, and chose the strategy that performed the best on average. In each tournament, I randomly sampled a large number of competing submissions based on various strategies and played all combinations in head-to-head matchups. Some strategies used uniform sampling across districts, while others used front- or back-loaded biased sampling. I also included several deterministically-chosen submissions that I thought would be popular choices. I scored submissions based on the scoring rules outlined in the game description.
\\

\textbf{Some low-level details:}

\begin{itemize}
\item Given that we are only allowed to place a whole number of resources in each district, some resource allocations are less ``wasteful'' than others, from an offensive-viewpoint. For example, placing 2 resources in district 10 will not win a vote in any situations where 1 resource will not. Furthermore, for district 10, the least-wasteful allocations are $\{0, 1, 11, 21, \ldots, 91\} = \{0\} \cup \{ \,k \in \mathbb{N} | k \mod 10 = 1 \land k \leq 100\}$.
Thus, since our objective is to play offensively, I made sure to simulate a number of submissions that followed this minimal-waste strategy. Ultimately, the submission that I chose also adheres to this strategy.
\item I simulated submissions from the Dirichlet-multinomial distribution. I chose the shape parameter vector of the Dirichlet distribution according to the type of strategy I wanted to simulate.
\end{itemize}

\end{homeworkProblem}


%----------------------------------------------------------------------------------------
%	PROBLEM 3
%----------------------------------------------------------------------------------------

\begin{homeworkProblem}[\arabic{homeworkProblemCounter}. Does it change if you are allowed decimal numbers of units?]

Yes, I would makes some changes to my approach. First, the idea of least-wasteful allocations (mentioned above) no longer applies. Second, I would need to alter the sampling distributions for my simulations, or simply add some random noise to each submission. 
\\

That said, I would still employ a similar approach of simulating tournaments of strategically-biased random submissions. With more time, I would construct a larger tournament of tournament-winners as a genetic algorithm. In this algorithm, a tournament is a generation of species (submissions). Species are scored based on a fitness function (the scoring rules), and propagated to the next generation with probability proportional to their fitness. This approach would help to better explore a larger space of possible submissions.


\end{homeworkProblem}


\end{document}
